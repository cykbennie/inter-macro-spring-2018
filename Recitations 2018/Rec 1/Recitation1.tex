\documentclass[12pt]{article} 

% Set page margins and size
\usepackage[papersize={8.5in,11in}, left=1in, top=1in, right=1in, bottom=1in]{geometry}

%\usepackage[landscape]{geometry}

\usepackage[english]{babel}         % Permite escribir en espanol
\usepackage[ansinew]{inputenc}      % Permite escribir en espanol
\usepackage{graphicx}               % Para insertar graficos
\usepackage{float, verbatim}        % Para manejar la ubicacion de graficos
\usepackage{verbatim}               % Para escribir codigos
\usepackage{psfrag}                 % Para escribir en los graficos
\usepackage{url}                    % Para escribir direcciones web
\usepackage{subfigure}                 % Para poner varias figuras en el mismo marco
\usepackage{hhline}                 % Para poner varias figuras en el mismo marco
\usepackage{longtable}              % Para que las tablas no se corten
\usepackage{cancel}
\usepackage{isorot}
\usepackage{caption}

\usepackage{amsmath}               
\usepackage{bm}
\usepackage{amsthm}                 
\usepackage{amssymb}                
\usepackage{lscape}

\usepackage{color}

% Hyperlinks
\usepackage{hyperref}


\parindent=0pt
% Increase paragraph spacing
\setlength{\parskip}{6pt}
\usepackage{setspace}

\usepackage{booktabs}
\usepackage[flushleft]{threeparttable}
%\usepackage{tablefootnote}

\newcommand{\bi}{\begin{itemize}}
\newcommand{\ei}{\end{itemize}}
\newcommand{\Rkplus}{\mathbb{R}^K_+}
\newcommand{\bee}{\begin{equation}}
\newcommand{\eee}{\end{equation}}

\newcommand{\be}{\begin{equation*}}
\newcommand{\ee}{\end{equation*}}


\newtheorem{mydef}{Definition}
\newtheorem{ex}{Example}
\newtheorem{lemma}{Lemma}[section]
\newtheorem{prop}{Proposition}
\newtheorem{corr}{Correlary}
\newtheorem{axiom}{Axiom}[section]
\newtheorem{thm}{Theorem}[section]
\newtheorem{remark}{Remark}[section]
%\pagenumbering{gobble}

\usepackage{pgf}
\usepackage{tikz}
\usetikzlibrary{trees,shapes,snakes,fit}

\newcommand\independent{\protect\mathpalette{\protect\independenT}{\perp}}
\def\independenT#1#2{\mathrel{\rlap{$#1#2$}\mkern2mu{#1#2}}}
\renewcommand{\vec}[1]{\mathbf{#1}}
\usepackage{enumerate}

\begin{document}
{\Large \bf Intermediate Macroeconomics Recitation 1}

Topics: partial derivatives, Taylor series approximations, and optimization.

%\tableofcontents
 \section{Partial derivatives}
 
 
 %These come up for the first time in the context of the firm's optimal choice of labor and capital. Please feel free to talk about that case (or the household problem) even though I will also do this in lecture. There are notes from prior years on Dropbox. But I have also added a reading about this to the syllabus. This is chapter 14 of Simon and Blume (1994). It may be useful to go through stuff from this reading.
 %References: Simon and Blume, Chapter 14
 
Partial derivatives look at the variation of a function $f(x_1, \dots, x_n)$ brought about by the change in only one variable, say $x_i$. This can be referred to as $\partial f / \partial x_i$, $f_{x_i}$, $f_i$, or $D_{i}f$.

\begin{mydef} Let $f:\mathbb{R}^n \rightarrow \mathbb{R}$. Then for each variable $x_i$ at each point $\mathbf{x}^0 = (x_1^0, \dots, x_n^0)$ in the domain of $f$, 
\begin{align*}
\frac{\partial f}{\partial x_i} (x_1^0, \dots, x_n^0) = \lim_{h \rightarrow 0} \frac{f(x_1^0, \dots, x_i^0+h, \dots, x_n^0) - f(x_1^0, \dots, x_i^0, \dots, x_n^0)}{h}
\end{align*}
if the limit exists. Only the $i$th variable changes; the others are treated as constants.
 \end{mydef}
  
 For example, in the case of
two arguments, we can write:
\begin{equation*}
z=F\left( x,y\right)
\end{equation*}%
A partial derivatives of $z$ with respect to $x$, denoted by $\partial
z/\partial x=\partial F(x,y)/\partial x$ measures the \textit{marginal change%
} in $z$ in response to a \textit{marginal change }in $x$ while $y$ is kept 
\textit{constant}. When computing partial derivatives, the usual rules from
calculus apply and you should know how to deal with the most common of them.

\begin{ex} Consider the function $f(x,y) = 3x^2 y^2 + 4xy^3 + 7y$. Compute the partial derivatives with respect to $x$ and $y$.
\end{ex}

In this course, the two most common types of functions we are going to see
are production functions and utility functions.

\subsection{Production functions}

A production function measures how much output $Y_{t}$ we get in period $t$ from combining some
amount of capital $K_{t}$ and labor $L_{t}$ with total factor productivity
(TFP) $A_{t}$. A general way to write this production function is: 
\begin{equation*}
Y_{t}=F\left( A_{t},K_{t},L_{t}\right),
\end{equation*}%
Commonly used production functions are the Cobb Douglas production function: 
\begin{align*}Y_t = A K_t^\alpha L_t^{1-\alpha}, \end{align*}
and the Constant Elasticity of Substitution (CES) production function:
\begin{align*}Y_t = A \left[ \pi K_t^\frac{\sigma-1}{\sigma} + (1-\pi) L_t^\frac{\sigma-1}{\sigma} \right]^\frac{\sigma}{\sigma-1}. %\tag{CES}
\end{align*}


For the production function, the partial derivative with respect to $K$ is called the {\bf marginal product of capital} (MPK). This is the rate at which output changes with respect to capital, with labor held fixed. Similarly, the {\bf marginal product of labor} (MPL) measures the rate at which output changes with respect to labor, with capital held fixed. 
%Mathematically, marginal product is the first derivative of the
%function $F$ with respect to its input, i.e. 
%\begin{equation*}
%MPL=\frac{\partial F}{\partial L} \text{ and }MPK=\frac{\partial F}{%
%\partial K}
%\end{equation*}

We can use derivatives to learn some properties about production functions.

\begin{ex} Compute the partial derivatives of the Cobb-Douglas production function.
\end{ex}

The partial derivatives of the Cobb-Douglas production function are both positive. This property is called {\bf positive marginal returns}. It means that as either input (capital or labor) increases, holding the other one constant, output increases. Mathematically, \begin{equation*}
MPL=\frac{\partial F}{\partial L}\geq 0\text{ and }MPK=\frac{\partial F}{%
\partial K}\geq 0
\end{equation*}

\begin{ex} Compute the second derivatives of the Cobb-Douglas production function.
\end{ex}

The second derivatives of the Cobb-Douglas production function are both negative. This property is called {\bf diminishing marginal returns}. It means that if we keep on increasing either input (capital or labor), holding the other one constant, output will continue to increase but by decreasing amounts. 
Mathematically, \ 
\begin{equation*}
\frac{\partial ^{2}F}{\partial K^{2}}<0\text{ and }\frac{\partial ^{2}F}{%
\partial L^{2}}<0
\end{equation*}


%Properties of production functions:
%\begin{itemize}
%\item Positive Marginal Returns:\ Given the same amount of labor input, if
%we increase physical input $K$, we get more output.
%
%\begin{itemize}
%\item For example, originally, we can produce 12 cookies from 1 oven and 1
%cook. Now, if we increase 1 more oven to a total of 2 ovens and 1 cook, we
%can produce 20 cookies. Notice that we have more cookies than before! We
%call this\textit{\ positive marginal products of capital} namely, given the
%same amount of labor input, more input of capital leads to more output.\
%Similarly for Marginal Products of Labor.
%
%\item Mathematically, marginal product is the first derivative of the
%function $F$ with respect to its input, i.e. 
%\begin{equation*}
%MPL=\frac{\partial F}{\partial L}\geq 0\text{ and }MPK=\frac{\partial F}{%
%\partial K}\geq 0
%\end{equation*}
%\end{itemize}

A third property of many production functions is that they are {\bf constant returns to scale}. This means that if you increase all the inputs {\it proportionally}, output increases by the same proportion. Mathematically, this means 
\begin{equation*}
Y= F(K,L)\Rightarrow F(2K,2L)=2Y
\end{equation*}%
We can triple the inputs $K$ and $L,$ we get triple the amount of output. In
general, we have%
\begin{equation*}
F(\lambda K,\lambda L,A)=\lambda F(K,L,A)\text{ for any }\lambda >0
\end{equation*}

 What is \textit{Increasing returns to scale (IRS)}? If we double the
physical and labor inputs, and use the same formula, we will have more than
twice the amount of cookies, i.e. 
\begin{equation*}
Y=F(K,L,A)\Rightarrow F(2K,2L,A)>2Y
\end{equation*}

 Similarly, \textit{Decreasing returns to scale (DRS)}? If we double
the physical and labor inputs, and use the same formula, we will have less
than twice the amount of cookies, i.e. 
\begin{equation*}
Y=F(K,L,A)\Rightarrow F(2K,2L,A)<2Y
\end{equation*}
%\end{itemize}
%
%
%\item Diminishing Marginal Returns: \textit{Diminishing marginal returns to
%capital}\ is defined as if we keep on increasing capital holding the level
%of labor input constant, the return will continue to grow but additional
%output you produce will be smaller.
%
%\begin{itemize}
%\item Intuition: We know that the more ovens we put in, the more cookies we
%get. But if you keep on adding oven, and you still have the same number of
%cook (1 cook), the number of \textit{extra }cookies you get decreases. In
%fact, if you add an extra oven to a total of 1,000 ovens and only 1 cook,
%compared to when you have 999 ovens and 1 cook, you get almost no extra
%cookies! This is what we call \textit{diminishing marginal returns to
%capital.}
%
%\item Similarly, if we increase 1 more cook to have 1000 cooks, and 1 oven,
%the increase in cookies compared to when we have 999 cooks and 1 oven is
%almost none. This means we have \textit{diminishing marginal returns to labor%
%}.
%
%\item Mathematically, \ 
%\begin{equation*}
%\frac{\partial ^{2}F}{\partial K^{2}}<0\text{ and }\frac{\partial ^{2}F}{%
%\partial H^{2}}<0
%\end{equation*}
%\end{itemize}
%
%\item Example:\ Find the $MPL$, $MPK$ and second derivatives for 
%\begin{eqnarray*}
%Y &=&AK^{\alpha }L^{\beta } \\
%Y &=&A\left( K^{\alpha }+L^{\beta }\right)
%\end{eqnarray*}
%\end{itemize}

\subsection{Utility function}

A utility function links a level of value an individual gets from
consuming different types of goods. For example:\ your utility from
consumption and leisure is $U\left( c,l\right) $. Then $\partial
U/\partial c$ is the marginal utility of consumption and $\partial
U/\partial l$ is the marginal utility of leisure.

%\begin{itemize}
%\item Example:\ 
%\begin{equation*}
%
%\end{equation*}%
%where $L$ is labor supply.
%\end{itemize}

\begin{ex} Compute the partial derivatives of the utility function $U=a\ln C+b\ln \left( L\right)$.
\end{ex}

\section{Taylor series approximations}

%We will make use of linear approximations a lot in the class. So, students will need to know what a Taylor series approximation is. I will talk about this in lecture. But please cover this in recitation as well. This will, in particular, play a big role when I talk about Calculus of Variations. Again, there are notes from prior years on Dropbox. But there is also a reading on the syllabus from Apostol�s book chapters 7.1 - 7.9.

Taylor series approximations are approximation of functions into a polynomial. Suppose we would like to
understand the properties of a function near $x_{0}$. A Taylor series approximation gives: 
%Taylor series expansion (TSE) gives us the following. 
\begin{equation*}
f(x)=f(x_{0})+f^{\prime }(x_{0})(x-x_{0})+\frac{1}{2}f^{\prime \prime
}(x_{0})(x-x_{0})^{2}+...
\end{equation*}%
Given that we have sufficient knowledge of the function at $x_{0}$ $%
\{f(x_{0}),f^{\prime }(x_{0}),f^{\prime \prime }(x_{0}),...\}$, we are also
able to study the movement of the function near $x_{0}$.

In general, for a polynomial of degree $n$:
\[P(x)=\sum_{k=0}^n \frac{f^{(k)}(x_{0})}{k!} (x-x_0)^k \]

Let $f(x)=log(1+x)$.
We know that $f(0)=log(1)=0$ and we want to study the properties of the
function near $0$. Recall that, 
\begin{equation*}
\frac{d}{dx}log(g(x))=\frac{g^{\prime }(x)}{g(x)}.
\end{equation*}%
Hence, $f^{\prime }(x)=\frac{1}{1+x}$ so that $f^{\prime }(0)=1$. Now,
instead of dealing with the exact form of $f(x)$ away from $x=0$, the Taylor expansion
gives the following:
\begin{align*}
f(x)& \approx f(0)+f^{\prime }(0)(x-0) \\
& =x
\end{align*}%
Therefore, we approximate $log(1+x)=x$ when $x$ is near $0$. How precise is
this approximation? 
\begin{align*}
log(1.02) &=0.0198 \\
log(1.1) &=0.0953 \\
log(1.2) &=0.1823 \\ 
log(1.5) &=0.4055 
\end{align*}

The approximation gets
poor when $x$ is no longer `close' to $0$. However, it does give a
relatively good approximation near $0$. Now, why are we particularly
interested in this function? It turns out that in many macroeconomic series,
a linear time trend seems to be the case. This can be verified if we plot
the log of the series over time. In particular, suppose that we plot log
real GDP over time. That is, we plot the y axis for log of real GDP and the
x axis for time. We have seen in class that this log scale (or ratio scale
as in Jones) produces a graph that has a strong linear time trend. It turns
out that the slope of this time trend is actually the growth rate of real
GDP. This again comes from the above first order TSE.

\begin{align*}
\text{slope} &= \frac{log(rGDP_{t})-log(rGDP_{t-1})}{t-(t-1)} \\
&= log \left(\frac{rGDP_{t}}{rGDP_{t-1}}\right) \\
&= log \left(\frac{rGDP_{t-1}+(rGDP_{t}-rGDP_{t-1})}{rGDP_{t-1}}\right) \\
&= log \left( 1+\text{GrowthRate}_{t}\right) \\
&= \text{GrowthRate}_{t}
\end{align*}
%
%\FRAME{ftbpF}{4.1589in}{3.4471in}{0pt}{}{}{Figure}{\special{language
%"Scientific Word";type "GRAPHIC";maintain-aspect-ratio TRUE;display
%"USEDEF";valid_file "T";width 4.1589in;height 3.4471in;depth
%0pt;original-width 6.9272in;original-height 5.7363in;cropleft "0";croptop
%"1";cropright "1";cropbottom "0";tempfilename
%'MHAHHU00.wmf';tempfile-properties "XPR";}}


\section{Optimization}

%Please talk about unconstrained optimization and constrained optimization where we have an equality constraint. When talking about constrained optimization, please "plug the constraint" into the objective function. DO NOT TALK ABOUT LAGRANGIANS. The recitation notes from some earlier years may talk about Lagrangians. But we don't need this. And it confuses some students. There will be enough math in the class. We should avoid introducing math that is not needed.



The standard modelling approach in macroeconomics starts from considering
how, for example, individuals and firms make decisions and then aggregating
these decisions, imposing some equilibrium conditions, to understand how
macro variables behave. We generally model the behavior of these single
entities through some optimization problem: consumer maximize their
happiness, firms maximize their profits, firms minimize their cost, etc.
There are some important assumptions when we write an optimization problem
and Jon has discussed some of them (example:\ firms maximization problem).
But optimization, both constrained and uncostrained, is an important toll
that you must be comfortable with. We are going to review the basic steps
using 2 examples:\ one unconstrained and one constrained.

%\begin{example}
\subsection{Unconstrained maximization}

\[Max \quad or \quad Min \quad f(x,y) \]
\begin{enumerate}
\item Take first order conditions (FOC).
\item Solve optimal choice of $x^*,y^*$.
\item Plug $x^*,y^*$ back to the objective function, get maximum or minimum value. 
\end{enumerate}

\begin{ex}Consider the profit maximization problem%
\begin{equation*}
\Pi =AK^{\alpha }L^{\beta }-wL-rK
\end{equation*}%
where $\alpha ,\beta <1$

Solution method:\ Take derivatives and equalize them to 0, we called this
first order conditions (FOCs)%
\begin{eqnarray*}
\frac{\partial \Pi }{\partial K} &=&A\alpha K^{\alpha -1}L^{\beta }-r=0 \\
\frac{\partial \Pi }{\partial L} &=&A\beta K^{\alpha }L^{\beta -1}-w=0
\end{eqnarray*}%
which means%
\begin{equation*}
MRTS=\frac{MPK}{MPL}=\frac{r}{w}=\frac{\alpha }{\beta }\frac{L}{K}
\end{equation*}%
is the marginal rate of transformation. Given $r$ and $w$, we can find $K$
and $L$ (when $\alpha + \beta < 1$)%
\begin{eqnarray*}
r &=&A\alpha K^{\alpha -1}\left( \frac{\beta }{\alpha }\frac{r}{w}K\right)
^{\beta } \\
&\Rightarrow &K=\left( A\alpha ^{1-\beta }\alpha ^{\beta }r^{\beta
-1}w^{-\beta }\right) ^{\frac{1}{1-\alpha -\beta }} \\
&\Rightarrow &L=\left( A\beta ^{1-\alpha }\alpha ^{\alpha }w^{\alpha
-1}r^{-\alpha }\right) ^{\frac{1}{1-\alpha -\beta }}
\end{eqnarray*}%
with $K$ and $L,$ we can plug back to profit function above to find the
maximized profit.


If $\alpha +\beta =1$, i.e. we have Cobb-Douglas Production function, then
we get 
\begin{eqnarray*}
\alpha Y &=&rK \\
\left( 1-\alpha \right) Y &=&wL
\end{eqnarray*}%
that means we have the share of labor (and capital) in total GDP is constant
over time. The share of capital is $\alpha $ and share of labor is $1-\alpha
.$ We can use data to find $\alpha$.

\end{ex}

%\begin{example}
\subsection{Constrained maximization}

\[Max \quad or \quad Min \quad f(x,y) \]
\[s.t. \quad g(x,y)=0 \]
\begin{enumerate}
\item Plug the constraint equation into the objective function.
\item Solve unconstrained optimization problem. 
\end{enumerate}

\begin{ex} Consider the utility function:
\begin{eqnarray*}
\max_{c, l} U\left( c,l\right)  &=&a\ln c+b\ln l \\
s.t. &:&pc=wH \\
     &:& l = 1-H \text{(We normalize the total amount of time to be 1)}
\end{eqnarray*}%
where $c$ is consumption, $l$ is leisure and $H$ is hours worked, $p$ is consumption goods
prices.

For simplicity, assume $p=1$. We can then substitute the constraints
into the objective function as follows:%
\begin{equation*}
\max_{c, l} a\ln c+b\ln l = \max_{H} a\ln (wH) +b\ln \left(
1-H\right) 
\end{equation*}%
As before, we take FOC\ w.r.t. $H$:%
\begin{eqnarray*}
\frac{aw}{wH} &=&\frac{b}{1-H}\Rightarrow \frac{\frac{b}{1-H}}{\frac{a}{c}}=w
\\
\frac{a}{H} &=&\frac{b}{1-H}\Rightarrow H=\frac{a}{a+b} \\
&\Rightarrow &c^{\ast }=\frac{wa}{a+b}
\end{eqnarray*}%
The condition that $\frac{\frac{b}{1-H}}{\frac{a}{c}}=w$ means the Marginal
rate of substitution between working and consumption is the wage. 
 \end{ex}


\end{document}
